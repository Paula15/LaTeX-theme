% !Mode:: "TeX:UTF-8"
\documentclass{article}
\usepackage[hyperref, UTF8]{ctex}
\usepackage[dvipsnames]{xcolor}
\usepackage{amsmath}
\usepackage{amssymb}
\usepackage{amsfonts}
\usepackage{listings}
\usepackage{pgfplotstable}
\usepackage{graphicx,float,wrapfig}
\usepackage{pgfplots}
\usepackage{fontspec}
\usepackage{hyperref}
\usepackage{booktabs} % 表格上的不同横线
\usepackage{ifthen}
\usepackage{titlesec}
\usepackage[square,sort,comma,numbers]{natbib}
\usepackage{url}
\usepackage{lstautogobble}
\usepackage{longtable}
\usepackage[top = 1in, bottom = 1in, left = 1in, right = 1in]{geometry}

\titleformat*{\section}{\sanhao \bf \centering}

\setmonofont[Mapping={}]{Monaco}  %英文引号之类的正常显示,相当于设置英文字体
\setsansfont{Monaco} %设置英文字体 Monaco, Consolas,  Fantasque Sans Mono

\newcommand{\image}[3][3in]{
    \begin{figure}[H]
        \centering\includegraphics[width=#1]{images/#2}
        \caption{#3}
    \end{figure}
}

\newcommand{\miniimage}[3][2.5in] {
    \begin{minipage}[H]{0.4\linewidth}
        \centering\includegraphics[width=#1]{images/#2}
        \caption{#3}
    \end{minipage}
}

\newcommand{\itembf}[1]{\item \textbf{#1}}

\newcommand{\tabletwo}[2] {
    \begin{table}[H] \centering
    \begin{tabular}{ c c }
    \toprule
    \textbf{#1} & \textbf{#2} \\
    \midrule
}

\newcommand{\tabletwoL}[2] {
    \begin{table}[H] \centering
    \begin{tabular}{ l l }
    \toprule
    \textbf{#1} & \textbf{#2} \\
    \midrule
}

\newcommand{\tablethree}[3] {
    \begin{table}[H] \centering
    \begin{tabular}{ c c c }
    \toprule
    \textbf{#1} & \textbf{#2} & \textbf{#3} \\
    \midrule
}

\newcommand{\tablethreeL}[3] {
    \begin{table}[H] \centering
    \begin{tabular}{ l l l }
    \toprule
    \textbf{#1} & \textbf{#2} & \textbf{#3} \\
    \midrule
}

\newcommand{\tablefour}[4] {
    \begin{table}[H] \centering
    \begin{tabular}{ c c c c }
    \toprule
    \textbf{#1} & \textbf{#2} & \textbf{#3} & \textbf{#4} \\
    \midrule
}

\newcommand{\tablefourL}[4] {
    \begin{table}[H] \centering
    \begin{tabular}{ l l l l }
    \toprule
    \textbf{#1} & \textbf{#2} & \textbf{#3} & \textbf{#4} \\
    \midrule
}

\newcommand{\regtable}[5] {
    \begin{table}[H] \centering
    \begin{tabular}{ l l p{8cm} l l }
    \toprule
    \textbf{#1} & \textbf{#2} & \textbf{#3} & \textbf{#4} & \textbf{#5} \\
    \midrule
}

\newcommand{\tablefive}[5] {
    \begin{table}[H] \centering
    \begin{tabular}{ c c c c c }
    \toprule
    \textbf{#1} & \textbf{#2} & \textbf{#3} & \textbf{#4} & \textbf{#5} \\
    \midrule
}

\newcommand{\tablefiveL}[5] {
    \begin{table}[H] \centering
    \begin{tabular}{ l l l l l }
    \toprule
    \textbf{#1} & \textbf{#2} & \textbf{#3} & \textbf{#4} & \textbf{#5} \\
    \midrule
}

\newcommand{\testtable}[5] {
    \begin{longtable}{ l p{3cm} p{8cm} p{2.2cm} p{2.2cm} }
    \toprule
    \textbf{#1} & \textbf{#2} & \textbf{#3} & \textbf{#4} & \textbf{#5} \\
    \midrule
}

\newcommand{\testtableend}[1] {
    \bottomrule
    \caption{#1}
    \end{longtable}
}

\newcommand{\longtableend} {
    \bottomrule
    \end{longtable}
}

\newcommand{\tablesix}[6] {
    \begin{table}[H] \centering
    \begin{tabular}{ c c c c c c }
    \toprule
    \textbf{#1} & \textbf{#2} & \textbf{#3} & \textbf{#4} & \textbf{#5} & \textbf{#6} \\
    \midrule
}

\newcommand{\tablesixL}[6] {
    \begin{table}[H] \centering
    \begin{tabular}{ l l l l l l }
    \toprule
    \textbf{#1} & \textbf{#2} & \textbf{#3} & \textbf{#4} & \textbf{#5} & \textbf{#6} \\
    \midrule
}

\newcommand{\tableend} {
    \bottomrule
    \end{tabular}
    \end{table}
}

\newcommand{\tablecapend}[1] {
    \bottomrule
    \end{tabular}
    \caption{#1}
    \end{table}
}

\newcommand{\chuhao}{\fontsize{42pt}{\baselineskip}\selectfont}
\newcommand{\xiaochuhao}{\fontsize{36pt}{\baselineskip}\selectfont}
\newcommand{\yihao}{\fontsize{28pt}{\baselineskip}\selectfont}
\newcommand{\erhao}{\fontsize{21pt}{\baselineskip}\selectfont}
\newcommand{\xiaoerhao}{\fontsize{18pt}{\baselineskip}\selectfont}
\newcommand{\sanhao}{\fontsize{15.75pt}{\baselineskip}\selectfont}
\newcommand{\sihao}{\fontsize{14pt}{\baselineskip}\selectfont}
\newcommand{\xiaosihao}{\fontsize{12pt}{\baselineskip}\selectfont}
\newcommand{\wuhao}{\fontsize{10.5pt}{\baselineskip}\selectfont}
\newcommand{\xiaowuhao}{\fontsize{9pt}{\baselineskip}\selectfont}
\newcommand{\liuhao}{\fontsize{7.875pt}{\baselineskip}\selectfont}
\newcommand{\shibahao}{\fontsize{18pt}{\baselineskip}\selectfont}
\newcommand{\shisihao}{\fontsize{14pt}{\baselineskip}\selectfont}
\newcommand{\qihao}{\fontsize{5.25pt}{\baselineskip}\selectfont}

\definecolor{mygreen}{rgb}{0,0.6,0}
\definecolor{mygray}{rgb}{0.5,0.5,0.5}
\definecolor{mymauve}{rgb}{0.58,0,0.82}
\lstset{ %
    % backgroundcolor=\color{white},   % choose the background color
    basicstyle=\footnotesize\ttfamily,        % size of fonts used for the code
    % columns=fullflexible,
    breaklines=true,                 % automatic line breaking only at whitespace
    % captionpos=b,                    % sets the caption-position to bottom
    tabsize=4,
    % backgroundcolor=\color[RGB]{245,245,244},            % 设定背景颜色
    commentstyle=\color{mygray},    % comment style
    % escapeinside={\%*}{*)},          % if you want to add LaTeX within your code
    % keywordstyle=\color{blue},       % keyword style
    % stringstyle=\color{mymauve}\ttfamily,     % string literal style
    showstringspaces=false,                % 不显示字符串中的空格
    % frame=none,
    % rulesepcolor=\color{red!20!green!20!blue!20},
    % identifierstyle=\color{red},
    rulecolor=\color{black},
    frame=bt,
    language=python,
    keepspaces=true,
    autogobble,
    captionpos=b,
    emphstyle=\bfseries\color{blue},
}

% 设置hyperlink的颜色
\newcommand\myshade{85}
\colorlet{mylinkcolor}{violet}
\colorlet{mycitecolor}{YellowOrange}
\colorlet{myurlcolor}{Aquamarine}

\hypersetup{
  linkcolor  = mylinkcolor!\myshade!black,
  citecolor  = mycitecolor!\myshade!black,
  urlcolor   = myurlcolor!\myshade!black,
  colorlinks = true,
}
